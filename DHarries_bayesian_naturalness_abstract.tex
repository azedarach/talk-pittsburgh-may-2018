\documentclass[12pt,a4paper]{article}

\usepackage[utf8x]{inputenc}
\usepackage[T1]{fontenc}
\usepackage{lmodern}

\usepackage{amsmath,amssymb}

\pagestyle{empty}

\begin{document}
\section*{Bayesian analysis and naturalness of (Next-to-)Minimal
  Supersymmetric Models}
In non-minimal supersymmetric (SUSY) models, additional tree-level
contributions to the Higgs mass provide a possible solution to the little
hierarchy problem of the minimal supersymmetric standard model (MSSM).  This
has generated increased interest in models such as the next-to-MSSM (NMSSM),
on the grounds that they may be more natural than the MSSM.  However,
traditional measures of fine-tuning do not provide a well-defined method for
making such comparisons, since the outcome depends heavily on the particular
definition of fine-tuning chosen.  We contrast the results of applying such
measures to the constrained MSSM and a semi-constrained NMSSM with those
obtained using so-called naturalness priors.  The latter arise automatically
in the context of a Bayesian analysis quantifying the plausibility that a
given model reproduces the weak scale.  Consequently, these naturalness priors
have a well-defined probabilistic interpretation, and allow naturalness to
be rigorously grounded in Bayesian statistics.  We find that results based
on naturalness priors agree qualitatively with the traditional measures of
fine-tuning used, and illustrate how naturalness priors can provide
valuable insight into the hierarchy problem.
\end{document}
